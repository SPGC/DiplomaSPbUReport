\section{Заключение}

В данной дипломной работе было описано решение задачи по выделению ледового канала на изображении с кормы ледокола, с использованием подходов с глубоким обучением и без него.
Главной задачей проекта было разработка алгоритма, способного выделить пиксельную маску ледового канала для упрощения его дальнейшего исследования и нахождения его 
характеристик, таких как ширина или скорость схождения.

В ходе выполнения проекта были разобраны современные методы сегментации изображений, был проведен анализ существующих решений в области глубокого обучения без учителя, 
а также рассмотрены классические (без глубокого обучения) подходы к сегментации. Анализ существующих подходов именно в этой сфере, помог нивелировать проблему отсутствия
размеченных датасетов арктических льдов и позволил провести обучение нейросети на относительно небольшом датасете.

Для решения задачи было реализовано два подхода: подход без глубокого обучения, чтобы установить его на судно до конца декабря 2023 года, чтобы уже плавающий ледокол мог собирать 
какие-то данные, второй подход использовал нейросеть для более точного обнаружения канала и установки на ледокол после его прибытия в порт. Оба подхода способны генерировать пиксельные маски канала. 
Проведенное тестирование сети показало, что предложенный метод способен эффективно выделять ледовый канал, однако результаты оказались не идеальными. 
Основными проблемами стали появление шумов и неточностей в маске, что связано с разнообразием условий съемки и сложностью ледовых структур.

Даже с текущими неточностями и ограничениями нейросетевой подход может и будет использован на практике. Он позволяет автоматизировать сбор и первичный анализ информации
о ледовом канале, облегчая работу команде ледокола. В дальнейшем планируется дорабатывать существующий алгоритм, как программно, улучшая нейросеть, так и технически, 
добавляя новые типы сенсоров.

Таким образом, результаты данной работы демонстрируют перспективность использования нейросетевых методов для сегментации ледового канала. Продолжение 
исследований в этом направлении, включая сбор более разнообразных данных и улучшение архитектуры модели, может привести к созданию более надежных и 
точных инструментов для навигации в арктических водах. Данный алгоритм в составе БИК будет установлен летом 2024 года на 4 ледокола серии Урал и на ледокол ``50 лет победы''.